\documentclass[addpoints]{exam}

\usepackage{hyperref}

% Header and footer.
\pagestyle{headandfoot}
\runningheadrule
\runningfootrule
\runningheader{CS 440}{HW 4}{Fall 2023}
\runningfooter{}{Page \thepage\ of \numpages}{}
\firstpageheader{}{}{}

\qformat{{\large\bf \thequestion. \thequestiontitle}\hfill}
\boxedpoints
% \printanswers

\title{Homework 5: Ray Tracer}
\author{CS 440 Computer Graphics\\Habib University}
\date{Fall 2023}

\begin{document}
\maketitle
\thispagestyle{empty}

In this assignment, you will implement an extendable ray tracing engine, and use it to render an original scene. Skeleton code is provided in the accompanying \texttt{raytracer} folder.

\begin{questions}

  \titledquestion{Basic Raytracer}
  We observe the  following raytracing setup. A \textit{sampler} uses a \textit{camera} to shoot \textit{ray}s through a \textit{viewplane}. Each ray traverses a scene which consists of \textit{geometry} whose appearance is determined through \emph{material}s. The ray returns some \textit{shading information} which is used to determine the shade to assign to a pixel in an \textit{image}.

The accompanying \texttt{raytacer} directory and its sub-directories contain classes modeling all of the above and more.
\begin{itemize}
\item The \texttt{geometry} folder contains declarations of an abstract class, \texttt{Geometry}, and concrete classes  \texttt{Plane}, \texttt{Sphere}, and \texttt{Triangle} which inherit from \texttt{Geometry}. You have to provide the required implementations.
\item The \texttt{materials} folder contains declarations of an abstract class, \texttt{Material}, and a concrete class  \texttt{Cosine} that inherits from \texttt{Material}. The cosine material assigns color based on the angle between the ray and the normal at the hit point. You have to provide the required implementations.
\item The \texttt{samplers} folder contains declarations of an abstract class, \texttt{Sampler}, and a concrete class  \texttt{Simple} that inherits from \texttt{Sampler}. The simple sampler samples each pixel once through its center. You have to provide the required implementations.
\item The \texttt{cameras} folder contains declarations of an abstract class, \texttt{Camera}, and concrete classes, \texttt{Parallel} and \texttt{Perspective}, that inherit from \texttt{Camera}. A parallel camera stores the direction of projection and a perspective camera stores the position of the center of projection. You have to provide the required implementations.
\end{itemize}

The \texttt{world} folder includes declarations of 2 classes. \texttt{ViewPlane} contains information on the view plane. \texttt{World} contains all the information required to render the scene--the geometry and associated materials, the view plane, the camera and sampler, and the background color. \texttt{World::build} populates the scene. You will reimplement this function each time to define a new image to be rendered. The \texttt{build} folder includes some sample implementations of \texttt{World::build}. You have to provide the required implementations.

The \texttt{utilities} folder includes declarations of utility classes, notably \texttt{Image} and \texttt{ShadeInfo}. \texttt{Image} holds pixel colors and writes an image to file in \href{https://en.wikipedia.org/wiki/Netpbm_format#PPM_example}{PPM format} (ASCII version). \texttt{ShadeInfo} contains all the information required for shading a point. These classes refer to classes from the \texttt{world} folder. Other useful classes in this folder are \texttt{Ray}, \texttt{Vector3D}, and \texttt{Point3D}. You have to provide the required implementations.

Everything comes together in \texttt{raytracer.cpp} in order to render a scene.

Go over the provided files to make sure that you understand the overall structure. Make sure to do so in a top-down manner. That is, start with \texttt{raytracer.cpp}, then look into the classes that it includes, then those that are included in these classes, and so on. At each stage, make sure to relate the content of the files with what you know about raytracing.

\textbf{Implement} the necessary classes such that running the  provided \texttt{raytracer.cpp} as-is renders the scene defined in \texttt{World::build}. You may use the 3 sample \texttt{build} functions provided in the \texttt{build} folder. Share your renders in the discussion thread for this homework.

  \titledquestion{Extending the Raytracer}

Now we will extend our ray tracing engine. There are many possibilities.

Object appearance may be enhanced by implementing new \emph{materials}. Anti-aliasing may be added by adding new \textit{samplers}. The scene may be lit by one or more \emph{light sources}. Speedup can be achieved through \textit{acceleration structures}. Secondary and shadow rays can be traced through various \textit{tracers}.

Correspondingly, new classes will have to be derived classes from \texttt{Material} and \texttt{Sampler}, and new hierarchies constructed for \texttt{Light}, \texttt{Acceleration}, and \texttt{Tracer}.

\textbf {Lighting} The light source in the previous question was at the camera. Implement other light sources, e.g. point, spotlight, and directional by adding a new folder called \texttt{lights} and populating it with a hierarchy of \texttt{Light} subclasses. Add a \texttt{std::vector<Light*>} member to \texttt{World} and use it to light the scene.

\textbf {Appearance} Implement new \texttt{Material} subclasses that apply more detailed shading that the \texttt{Cosine} shader.

  \textbf{Acceleration} Add a new folder called \texttt{acceleration} and populate it with a hierarchy of acceleration structures. Add an \texttt{Acceleration*} member to \texttt{World} and use it to compute ray intersections.

  \textbf{Ray Casting} The ray tracer in \texttt{raytracer.cpp} is very basic--it shades based on primary rays only. Add a new folder called \texttt{tracers} and populate it with a hierarchy of \texttt{Tracer} classes. You can start the hierarchy by moving the ray tracer in \texttt{raytracer.cpp} to a \texttt{Basic} class derived from \texttt{Tracer}. Add other ray tracers that implement other ray tracing features like shadows, recursive levels of reflection, and transparency. Add a \texttt{Tracer*} member to \texttt{World} and use it for ray tracing.

  \titledquestion{Showcase}
  We now want to showcase our ray tracing engine--how good it is and how it creates stunning images. A resource page is with helpful links and information is provided on Canvas.

  \textbf{Create an original scene} The scene should be \emph{original}. You may derive inspiration from past rendering competitions such as those listed in the resource page, but the final scene should be the product of your own imagination. You may use third-party assets that are publicly available for free. You can use them to build your original scene but it is not allowed for the whole, or major part, of the scene to be reused from somewhere. Alternatively, you can model everything yourself. Links to candidate scenes, 3D model repositories, and 3D modeling software are included in the resource page.

  \textbf{Use your engine to render the scene} You must create two renders of your scene: a low quality render at a resolution of 480x360 and a high quality render with resolution 1920x1080 or higher. You may use different aspect ratios if they better fit your scene. The images must be rendered by your engine. Any post-processing must be implemented within your framework. The images need not be realistic. You can use features that do not follow real world physics if it better suits your artistic concept. The high quality image must render in less than 6 hours on a modern computer.

  \textbf{Create a web-page to showcase your work} The web-page should feature your render and its title. It should include the following.
    \begin{itemize}
    \item \textbf{Concept} description of your concept and how you arrived at it.
    \item \textbf{Scene} describe how you built your scene.
    \item \textbf{Image Features} highlight interesting parts or features of your render. Additional images may be included for this purpose.
    \item \textbf{Code Features} list all the features that you have implemented in your ray tracer.
    \item \textbf{Acceleration} include a table comparing rendering times of your ray tracer with and without an acceleration structure. Supporting renderings must be included.
    \item \textbf{Build} for every image included on the page, a link to the corresponding implementation of {\tt World::build},
    \item \textbf{Acknowledgment} acknowledge all third party sources of used assets or resources, once where they are used, e.g. in the caption of a rendered image, and again in an \textit{Acknowledgment} section toward the bottom of the page.
    \item \textbf{Team} include the names of all team members and a photograph of your team.
    \item \textbf{Comments} include any other comments desired by the team.
    \end{itemize}

  Some sample web-pages are provided in the resource page. Any accompanying build or code files have been removed from the samples. As your submission may also be shared in the future as a sample, take care to only include images which you are comfortable sharing publicly.
  
\end{questions}

\section*{Credits}

This homework is adapted from the rendering competition run by \href{https://graphics.cg.uni-saarland.de/people/slusallek.html}{Philipp Slusallek} in his \href{https://graphics.cg.uni-saarland.de/courses/cg1-2020/}{Computer Graphics 1 course}. The code is adapted from that provided by \href{http://www.raytracegroundup.com/}{Kevin Suffern}.

\end{document}

%%% Local Variables:
%%% mode: latex
%%% TeX-master: t
%%% End: